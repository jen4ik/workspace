%%This is a very basic article template.
%%There is just one section and two subsections.
\documentclass{article}

\begin{document}

\begin{center} \LARGE\textbf{CMPSCI 611 - Advanced Algorithms}
\Large\textbf{Homework 1}
\vspace{0.5cm}\\Jennie Steshenko \small{(Collaboration: Emma Strubell)}
\end{center}

\section*{Question 1}
\subsection*{Part A}
The order of growth of functions is as follows
\\$g_7 = (n^{1/log_2 n})^3 = 8$, 
\\$g_{10} = (log n)^2$,
\\$g_1 = 2^{\sqrt{log_2 n}}$,
\\$g_3 = n(log_2 n)^3$,
\\$g_9 = n^{1+1/log logn}\approx{n^1.2}$,
\\$g_4 = n^{4/3}$,
\\$g_{11} = (logn)^{log n}$,
\\$g_5 = n^{log_2 n}$,
\\$g_6 = 2^n$,
\\$g_{12} = \pi^n = 3.14^n$,
\\$g_8 = n!$,
\\$g_2 = 2^{2^n}$

\subsection*{Part B}
\begin{enumerate}
  \item $log {f(n)} \; is \; O(log {g(n)})$ - \textbf{True}
    \\ $\; O(log(g(n))) = log {(cf(n))} = log c + log {f(n)}$
	\\ $\lim_{x\to\infty}\frac{log c + log {f(n)}}{log {f(n)}} =
	    \lim_{x\to\infty}(\frac{log c}{log {f(n)}} + \frac{log {f(n)}}{log
	    {f(n)}}) = 0 + 1 = 1$
  \item $2^{f(n)} \; is \; O(2^{g(n)})$ - \textbf{False}
    \\ $\; $ Let us take $\; f(n) = x \;$ and $\; g(n) = 2x \;$ then:
    \\ $\lim_{x\to\infty}\frac{2^{2x}}{2^x} \; =
        \lim_{x\to\infty}\frac{2^x*2^x}{2^x} \; =
        \lim_{x\to\infty}{2^x} \; = \infty$
  \item $(f(n))^2 \; is \; O((g(n))^2)$ - \textbf{True}
    \\ $\; O((g(n))^2) = (cf(n))^2 = c^2(f(n))^2$
    \\ $\lim_{x\to\infty}\frac{c^2(f(n))^2}{(f(n))^2} = c^2$
\end{enumerate}
\vfill
\subsection*{Question 2}
\subsubsection*{Proposed Algorithm}
The assumptions and notations used in this algorithm:
\begin{itemize}
  \item Arrays: A, B
  \item Array sizes: sizeA, sizeB (e.g. the index of the last cell in array A
  that should be checked)
\end{itemize}

\raggedright The Algorithm: The general idea is to look for the $K^{th}$
smallest item by scanning the arrays intermintently, in a manner similar to a
binary search.\\
\vfill

\subsection*{Question 3}
The algorithm to solve this problem is very similar to the MergeSort algorithm,
with one additional variable to count the number of inverted pairs, and one
additional line in the algorithm to track the value of the variable.\\
\textit{The suggested algorithm is a modification to an array based algorithm
of the MergeSort algorithm on Wikipedia.org}\\

\vspace{0.5cm} \texttt{function CountInvertedPairs(intArray intArr, int
start, int end)}\\
\texttt{// if list size is 1\, consider it sorted and return it}\\
\texttt{if sizeOf(intArr) <= 1}\\
\texttt{$\quad$    return intArr}\\
\texttt{// else array size is > 1, so split the array into two subarrays}\\
\texttt{// The counter of inverted pairs, as global}\\
\texttt{var integer invertedCount = 0}\\
\texttt{// The start and end of the left of the subarray }\\
\texttt{var int leftStart, leftEnd }\\
\texttt{// The start and end of the right of the subarray }\\
\texttt{var int rightStart, rightEnd }\\
\texttt{var integer middle = floor((end - start) / 2)}\\
\texttt{// recursively call CountInvertedPairs() to further split each}\\
\texttt{// sublist until subarray size is 1}\\
\texttt{left = CountInvertedPairs(intArr, start, middle)}\\
\texttt{right = CountInvertedPairs(intArr, middle + 1, end)}\\
\texttt{// merge the sublists returned from prior calls to}\\
\texttt{// CountInvertedPairs() and return the resulting merged sublist}\\
\texttt{return merge(left\, right\, invertedCount)}\\

\vspace{0.5cm} \texttt{function merge(intArray left, intArray right,
invertedCount)}\\
\texttt{var intArray result}\\
\texttt{// Indexes to each of the arrays - left, right and result}\\
\texttt{var int leftIndex = 0, rightIndex = 0, resIndex = 0}\\
\texttt{while sizeOf(left) > 0 or sizeOf(right) > 0}\\
\texttt{$\quad$ if sizeOf(left) > 0 and sizeOf(right) > 0}\\
\texttt{$\quad\quad$ if left[leftIndex] <= right[rightIndex]}\\
\texttt{$\quad\quad\quad$ result[resIndex] = left[leftIndex]}\\
\texttt{$\quad\quad\quad$ leftIndex++}\\
\texttt{$\quad\quad\quad$ resIndex++}\\
\texttt{$\quad\quad$ else}\\
\texttt{$\quad\quad\quad$ result[resIndex] = right[rightindex]}\\
\texttt{$\quad\quad\quad$ rightIndex++}\\
\texttt{$\quad\quad\quad$ resIndex++}\\
\texttt{$\quad\quad\quad$ invertedCount = rightIndex - resIndex}\\
\texttt{$\quad$ else if sizeOf(left) > 0}\\
\texttt{$\quad\quad$ result[resIndex] = left[leftIndex]}\\
\texttt{$\quad\quad$ leftIndex++}\\
\texttt{$\quad\quad$ resIndex++}\\
\texttt{$\quad$ else if sizeOf(right) > 0}\\
\texttt{$\quad\quad$ result[resIndex] = right[rightindex]}\\
\texttt{$\quad\quad$ rightIndex++}\\
\texttt{$\quad\quad$ resIndex++}\\
\texttt{$\quad\quad$ invertedCount = rightIndex - resIndex}\\
\texttt{end while}\\
\texttt{return result}\\
\vfill

\subsection*{Question 4} Assumption:
\begin{itemize}
  \item The set S is stored in an array
  \item All the numbers in the given set are less than \textit{x}. If not, a
  simple modification to the MergeSort algorithm can exclude all irrelevant
  numbers.
  \item None of the numbers in the set appears more than one. If not, a simple
  modification to the MergeSort algorithm can exclude all irrelevant numbers.
\end{itemize}

The algorithm:
\begin{itemize}
  \item Run arrayMergeSort(set S) - complexity of $O(n log{n})$\\
  \textit{(The same algorithm as presented in the solution to Question 3, without the counter and operations for counting
  inverted pairs)}
  \item Use a binary search to find \textit{half} - all members of the
  array to the left of this number are smaller or equal to $x/2$, and the
  members to the right of the array are bigger or equal to $x/2$ - complexity of
  $O(log(n))$
  \item If $indexOf(half, completeArray) < (sizeOf(completeArray))/2$ then\\
  $\quad$ matchToArr[start] = 0\\
  $\quad$ matchToArr[end] = indexOf(half, completeArray)\\
  $\quad$ matchFromArr[start] = indexOf(half, completeArray) + 1\\
  $\quad$ matchFromArr[end] =   sizeOf(completeArray)\\
  Else\\
  $\quad$ matchToArr[start] = indexOf(half, completeArray) + 1\\
  $\quad$ matchToArr[end] =   sizeOf(completeArray)\\
  $\quad$ matchFromArr[start] = 0\\
  $\quad$ matchFromArr[end] = indexOf(half, completeArray)\\
  \item For Each Item in matchToArray - complexity of $O(n log(n))$
  \begin{itemize}
    \item Use binarySearch in the matchFromArray to find the value $x - Item$
    \item If binarySearch returns NOT NULL\\
    $\quad$ Return TRUE and break   
    \item Return FALSE
  \end{itemize} 
\end{itemize}

The highest comlexity is encountered when the sorting of the complete set is
executed. The sum  matching process will run $m log(k)$ times, where at most $m
= k = n/2$, and in most cases $m < n/2$, hence the total matching process will
run faster than $O((n/2) log (n/2))$

\vfill

\subsection*{Question 5}
\indent The algorithm used to build spanning trees with minimal weight is
Kruskal's algorithm, which is greedy. At each iteration, the algorithm chooses
the next best candidate, in this case, the next edge with minimal weight that is
available for choosing.
 
\indent In order to have two different trees with the same minimal wight, the
algorithm must be able to chose 2 different subsets of edges that have the same sum
of weights. Since the algoritm is greedy, that can happen only in the case when
there are 2 edges with the same weight, that lead to different nodes, thus
allowing the existence of different final sets when the algorithm has finished
running. 

\indent However, it is indicated that all edges in the given set are of
different wieght, thus every time the algoritm will run it will make exactlyl
the same choices, thus each time the exact same minimal spanning tree will be
built.

\end{document}
