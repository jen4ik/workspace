%%This is a very basic article template.
%%There is just one section and two subsections.
\documentclass{article}

\begin{document}

\begin{center} \LARGE\textbf{CMPSCI 611 - Advanced Algorithms}
\Large\textbf{Homework 1}
\vspace{0.5cm}\\Jennie Steshenko \small{(Collaboration: Emma Strubell)}
\end{center}

\section*{Question 1}
\subsection*{Part A}
The order of growth of functions is as follows\\ 
$ g_7 = (n^{1/\log_2 n})^3 = (n^{{\log_2 n}/{\log_n n}})^3 = (2^{{\log_n
n}/{\log_n n}})^3 = 2^3 = 8,$\\
$g_{10} = (\log_2 n)^2,$\\
$g_1 = 2^{\sqrt{\log_2 n}},$\\
$g_3 = n(\log_2 n)^3,$\\
$g_9 = n^{1+1/\log_2 \log_2 n} = n \times {n^{1+1/\log_2 \log_2 n}},$\\ 
$g_4 = n^{4/3} = n^{1+1/3} = n \times n^{1/3},$\\
$g_{11} = (\log_2 n)^{\log_2 n},$\\
$g_5 = n^{\log_2 n},$\\
$g_6 = 2^n,$\\
$g_{12} = \pi^n = 3.14^n,$\\
$g_8 = n! \approx{n^n},$\\
$g_2 = 2^{2^n}$\\

Explanations for the ordering\\

$g_{10} = O(g_1):\\
\lim_{n\to\infty}\frac{\log^2n}{2^{\sqrt{\log n}}} =
\lim_{n\to\infty}\frac{2\log n}{2^{\sqrt{\log n}}} =
2\lim_{n\to\infty}\frac{\log\log n}{\log2^{\sqrt{\log n}}} = 
2\lim_{n\to\infty}\frac{\log\log n}{\sqrt{\log n}} =  
2\lim_{n\to\infty}\frac{\log\log\log n}{\log\log^{1/2}n} =
2\lim_{n\to\infty}\frac{\log\log\log n}{1/2\log\log n} = 
4\lim_{n\to\infty}\frac{\log\log\log n}{1\log\log n} = 0$\\
$\log\log\log n$ grows slower than $\log\log n$, hence the ratio test gives 0\\

$g1 = O(g3):\\
\lim_{n\to\infty}\frac{g_1}{g_3} = \lim_{n\to\infty}\frac{2^{\sqrt{\log_2
n}}}{n\log_2^3n} = \lim_{n\to\infty}\frac{\sqrt{\log_2 n}}{\log_2 (n\log^3n)}
= \lim_{n\to\infty}\frac{\sqrt{\log_2 n}}{\log_2 n + \log_2\log^3n} = 
\lim_{n\to\infty}\frac{\sqrt{\log_2 n}}{\log_2 n + 3\log_2\log n} = 0$\\
$\log n$ grows faster than $\sqrt{\log n}$, hence the ratio test
gives 0, and $2^{\sqrt{\log n}} \in o(n\log^3n)$\\

\vfill

$g_3 = O(g_9):\\
n^{\frac{1}{\log_2 \log_2 n}} \, ? \, (\log_2 n)^3\\
\frac{1}{\log_2 \log_2 n} \, ? \, \log_n (\log_2 n)^3 = \frac{\log_2(\log_2
n)^2}{\log_2 n} = \frac{3 \log_2(\log_2 n)}{\log_2 n}\\
\frac{1}{\log_2 \log_2 n} \, ? \, \frac{3\log_2 \log_2 n}{\log_2 n}\\
\log_2 n \, ? \, 3(\log_2\log_2 n)^2\\
\sqrt{\log_2 n} \, ? \, \sqrt{3}\sqrt{(\log_2 \log_2 n)^2}\\
\sqrt{\log_2 n} \, ? \, \sqrt{3} \log_2 \log_2 n; \, \{\log_2n = x \}\\
\lim_{x\to\infty}\frac{\sqrt{3}\log_2 x}{\sqrt{x}} = 0$\\

$g_9 = O(g_4):\\
\lim_{x\to\infty}\frac{1}{\log_2 \log_2 n} = 0\\ 
g_4 $ will be bigger than $g_9 \; \forall n>n_0.(\frac{1}{\log_2 \log_2\, n_0} =
1/3)$\\

$g4 = O(g11):\\
\lim_{n\to\infty}\frac{n^{4/3}}{\log n^{\log n}} = \lim_{n\to\infty}
\frac{\log n^{4/3}}{\log\log n^{\log n}} = \lim_{n\to\infty}\frac{4/3\log n}
{\log^2n}=\lim_{n\to\infty}\frac{4/3}{\log n} = 0$\\

$g_{11} = O(g_5):\\
\log n^{\log n} = \overbrace {\log n \times \log n \times ... \times \log
n}^{k = \log n \mathrm{\ times}}\\
n^{\log n} = \overbrace {n \times n \times ... \times n}^{k = \log n \mathrm{\
times}}$
Since $\log n \in O(n)$, clearly $\log n^{\log n} \in O(n^{\log n})$\\

$g_5 = O(g_6):\\
\lim_{n\to\infty}\frac{n^{\log n}}{2^n} = 
\lim_{n\to\infty}\frac{\log n^{\log n}}{n} = 
\lim_{n\to\infty}\frac{\log_n n^{\log n}/\log_n2}{n} =
\lim_{n\to\infty}\frac{\log n}{\log_n2\cdot n} = 0$\\
Because $\lg n \in O(n)$, the ratio test gives 0 and $n^{\lg n} \in o(2^n)$.

$g_{12} = O(g_8):\\
\pi^{n} = \overbrace{\pi \times \pi \times ... \times \pi}^{n \mathrm{\times}}\\
n! = 1 \times 2 \times 3 \times 4 \times ... \times n$


g8 = O(g2) (i think the change-of-base stuff is correct in this one, have
to go back and re-do the others):

We know by Stirling's approximation that $n! = O(n^n)$. So, it is
sufficient to show that $n^n \in O(2^{2^n})\\
\lim_{n \to \infty}\frac{n^n}{2^{2^n}} = \lim_{n \to \infty}\frac{\lg
n^n}{2^n} = \lim_{n \to \infty}\frac{\log_n n^n/\log_n2}{2^n} = \lim_{n \to
\infty}\frac{n}{\log_n2\cdot2^n} = 0$

Because $\log_n2$ is a constant and $n \in O(2^n)$, the ratio test gives
zero and $n^n \in o(2^{2^n})$, which implies that $n! \in O(2^{2^n})$

We know by Stirling's approximation that $n! = O(n^n)$. So, it is
sufficient to show that $n^n \in O(2^{2^n})$.
$
\lim_{n \to \infty}\frac{n^n}{2^{2^n}} = \lim_{n \to \infty}\frac{\lg
n^n}{2^n} = \lim_{n \to \infty}\frac{\log_n n^n/\log_n2}{2^n} = \lim_{n \to
\infty}\frac{n}{\log_n2\cdot2^n} = 0;$
 \\ Because $\log_n 2$ is a constant and $n \in O(2^n)$, the ratio test gives
 zero and $n^n \in o(2^{2^n})$, which implies that $n! \in O(2^{2^n})$

\subsection*{Part B}
\begin{enumerate}
  \item $log {f(n)} \; is \; O(log {g(n)})$ - \textbf{True}
    \\ $\; O(log(g(n))) = log {(cf(n))} = log c + log {f(n)}$
	\\ $\lim_{x\to\infty}\frac{log c + log {f(n)}}{log {f(n)}} =
	    \lim_{x\to\infty}(\frac{log c}{log {f(n)}} + \frac{log {f(n)}}{log
	    {f(n)}}) = 0 + 1 = 1$
  \item $2^{f(n)} \; is \; O(2^{g(n)})$ - \textbf{False}
    \\ $\; $ Let us take $\; f(n) = x \;$ and $\; g(n) = 2x \;$ then:
    \\ $\lim_{x\to\infty}\frac{2x}{x} = lim_{x\to\infty}2\frac{x}{x} = 2$
    \\ $\lim_{x\to\infty}\frac{2^{2x}}{2^x} \; =
        \lim_{x\to\infty}\frac{2^x*2^x}{2^x} \; =
        \lim_{x\to\infty}{2^x} \; = \infty$
  \item $(f(n))^2 \; is \; O((g(n))^2)$ - \textbf{True}
    \\ $\; O((g(n))^2) = (cf(n))^2 = c^2(f(n))^2$
    \\ $\lim_{x\to\infty}\frac{c^2(f(n))^2}{(f(n))^2} = c^2$
\end{enumerate}
\vfill
\subsection*{Question 2}
\subsubsection*{Proposed Algorithm}
The assumptions and notations used in this algorithm:
\begin{itemize}
  \item Arrays: A, B
  \item Array sizes: sizeA, sizeB (e.g. the index of the last cell in array A
  that should be checked)
\end{itemize}

\raggedright The Algorithm: The general idea is to look for the $K^{th}$
smallest item by scanning the arrays intermintently, in a manner similar to a
binary search.\\
\vfill

\subsection*{Question 3}
The algorithm to solve this problem is very similar to the MergeSort algorithm,
with one additional variable to count the number of inverted pairs, and one
additional line in the algorithm to track the value of the variable.\\
\textit{The suggested algorithm is a modification to an array based algorithm
of the MergeSort algorithm on Wikipedia.org}\\

\vspace{0.5cm} \texttt{function CountInvertedPairs(intArray intArr, int
start, int end)}\\
\texttt{// if list size is 1\, consider it sorted and return it}\\
\texttt{if sizeOf(intArr) <= 1}\\
\texttt{$\quad$    return intArr}\\
\texttt{// else array size is > 1, so split the array into two subarrays}\\
\texttt{// The counter of inverted pairs, as global}\\
\texttt{var integer invertedCount = 0}\\
\texttt{// The start and end of the left of the subarray }\\
\texttt{var int leftStart, leftEnd }\\
\texttt{// The start and end of the right of the subarray }\\
\texttt{var int rightStart, rightEnd }\\
\texttt{var integer middle = floor((end - start) / 2)}\\
\texttt{// recursively call CountInvertedPairs() to further split each}\\
\texttt{// sublist until subarray size is 1}\\
\texttt{left = CountInvertedPairs(intArr, start, middle)}\\
\texttt{right = CountInvertedPairs(intArr, middle + 1, end)}\\
\texttt{// merge the sublists returned from prior calls to}\\
\texttt{// CountInvertedPairs() and return the resulting merged sublist}\\
\texttt{return merge(left\, right\, invertedCount)}\\

\vspace{0.5cm} \texttt{function merge(intArray left, intArray right,
invertedCount)}\\
\texttt{var intArray result}\\
\texttt{// Indexes to each of the arrays - left, right and result}\\
\texttt{var int leftIndex = 0, rightIndex = 0, resIndex = 0}\\
\texttt{while sizeOf(left) > 0 or sizeOf(right) > 0}\\
\texttt{$\quad$ if sizeOf(left) > 0 and sizeOf(right) > 0}\\
\texttt{$\quad\quad$ if left[leftIndex] <= right[rightIndex]}\\
\texttt{$\quad\quad\quad$ result[resIndex] = left[leftIndex]}\\
\texttt{$\quad\quad\quad$ leftIndex++}\\
\texttt{$\quad\quad\quad$ resIndex++}\\
\texttt{$\quad\quad$ else}\\
\texttt{$\quad\quad\quad$ result[resIndex] = right[rightindex]}\\
\texttt{$\quad\quad\quad$ rightIndex++}\\
\texttt{$\quad\quad\quad$ resIndex++}\\
\texttt{$\quad\quad\quad$ invertedCount = rightIndex - resIndex}\\
\texttt{$\quad$ else if sizeOf(left) > 0}\\
\texttt{$\quad\quad$ result[resIndex] = left[leftIndex]}\\
\texttt{$\quad\quad$ leftIndex++}\\
\texttt{$\quad\quad$ resIndex++}\\
\texttt{$\quad$ else if sizeOf(right) > 0}\\
\texttt{$\quad\quad$ result[resIndex] = right[rightindex]}\\
\texttt{$\quad\quad$ rightIndex++}\\
\texttt{$\quad\quad$ resIndex++}\\
\texttt{$\quad\quad$ invertedCount = rightIndex - resIndex}\\
\texttt{end while}\\
\texttt{return result}\\
\vfill

\subsection*{Question 4} Assumption:
\begin{itemize}
  \item The set S is stored in an array
  \item All the numbers in the given set are less than \textit{x}. If not, a
  simple modification to the MergeSort algorithm can exclude all irrelevant
  numbers.
  \item None of the numbers in the set appears more than one. If not, a simple
  modification to the MergeSort algorithm can exclude all irrelevant numbers.
\end{itemize}

The algorithm:
\begin{itemize}
  \item Run arrayMergeSort(set S) - complexity of $O(n log{n})$\\
  \textit{(The same algorithm as presented in the solution to Question 3, without the counter and operations for counting
  inverted pairs)}
  \item Use a binary search to find \textit{half} - all members of the
  array to the left of this number are smaller or equal to $x/2$, and the
  members to the right of the array are bigger or equal to $x/2$ - complexity of
  $O(log(n))$
  \item If $indexOf(half, completeArray) < (sizeOf(completeArray))/2$ then\\
  $\quad$ matchToArr[start] = 0\\
  $\quad$ matchToArr[end] = indexOf(half, completeArray)\\
  $\quad$ matchFromArr[start] = indexOf(half, completeArray) + 1\\
  $\quad$ matchFromArr[end] =   sizeOf(completeArray)\\
  Else\\
  $\quad$ matchToArr[start] = indexOf(half, completeArray) + 1\\
  $\quad$ matchToArr[end] =   sizeOf(completeArray)\\
  $\quad$ matchFromArr[start] = 0\\
  $\quad$ matchFromArr[end] = indexOf(half, completeArray)\\
  \item For Each Item in matchToArray - complexity of $O(n log(n))$
  \begin{itemize}
    \item Use binarySearch in the matchFromArray to find the value $x - Item$
    \item If binarySearch returns NOT NULL\\
    $\quad$ Return TRUE and break   
    \item Return FALSE
  \end{itemize} 
\end{itemize}

The highest comlexity is encountered when the sorting of the complete set is
executed. The sum  matching process will run $m log(k)$ times, where at most $m
= k = n/2$, and in most cases $m < n/2$, hence the total matching process will
run faster than $O((n/2) log (n/2))$

\vfill

\subsection*{Question 5}
This is a simple proof by induction.
For n = 1: There is only one edge, hence it's possible to construct only one
minimal spanning tree\\
For n = k - 1: Assume that (k-1) edges were chosen by now, and they resulted
only in one minimal spanning tree\\
For n = k: An additional edge needs to be chosen at this point. Kruskal's
algorithm is a greedy algorithm and it always chooses the edge with the minimal
weight available. It is known that there are no to edges with the same weight,
hence when coming to choose, the group of edges with minimal weight is of size
1, and only that edge will be added to the minimal spanning tree.
This will repeat for each k, $0 < k < n$ according to the problem statement.

\end{document}
