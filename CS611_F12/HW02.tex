%%This is a very basic article template.
%%There is just one section and two subsections.
\documentclass{article}

\begin{document}

\begin{center} \LARGE\textbf{CMPSCI 611 - Advanced Algorithms}
\Large\textbf{Homework 1}
\vspace{0.5cm}\\Jennie Steshenko\\
\small{(Collaboration: Emma Strubell, Patrick Vergas)}\\
Estimated amount of hours invested in the assignment: 30
\end{center}

\section*{Question 1}


\subsection*{Question 2}
\subsubsection*{Proposed Algorithm}

\subsection*{Question 3}

For (E, I) to be a matroid, it has to satisfy the exchange property and the cardinality property.

\textbf{The Exchange Property:} According to the problem statement:\\
For $A, B \in I\:$\\
$\mid A \mid = k \Rightarrow \sum_{i=1}^N \left( \mid A \cap E_i \mid = 1 \right) = k$,\\ 
$\mid B \mid = j < k \Rightarrow \sum_{i=1}^N \left( \mid B \cap E_i \mid = 1 \right) < k $\\
Thus, $\exists e \in E_i \cdot \left(  A \cap E_i = \{ e \} \wedge B \cap E_i = \phi \right)$\\
Thus $B \cup \{e \} \in I$\\
Thus the Exchange Property is maintained

\textbf{The Cardinality Property:} According to the problem statement:\\
For $A, B \in I$, $\mid A \mid = \mid B \mid = \mid I \mid = k$\\
$\sum_{i=1}^N \left( \mid A \cap E_i \mid = 1 \right) = 
\sum_{i=1}^N \left( \mid B \cap E_i \mid = 1 \right)= k$\\
Thus, $\neg\exists E_i \in I \cdot \left( A \cap E_i = \phi \vee B \cap E_i = \phi \right)$\\
Thus A and B are maximal and of equal size, and maintain the Cardinality property.

\subsection*{Question 4}


\clearpage

\subsection*{Question 5}

\subsection*{Question 6}
\end{document}
